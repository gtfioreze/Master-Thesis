\section{Motivation}
\label{sec:int_motivation}

The understanding of visual perception and its processes is undertaken as one of the major modern challenges in the field of Neurosciences. In particular, neurons in visual pathways exhibit an interesting not fully characterized phenomenon: surround modulation, the effect of producing an altered response, when a stimulus is shown both in the cell's receptive field and in its surround. Involving Neurosciences, Physics and Engineering, we propose an extensive study on the effect of moving stimuli with different spatial location's in transgenic mice's field of view, onto neurons of the visual cortex, V1.  

The mapping of moving gratings properties and the spatial structure of surround modulation can produce a set of functional rules that imperate in the brain. It shall further provide insight into feedback mechanisms and the perception of visual scenes. Precluding the project, in here we review the main related theoretical, experimental and computational topics.

Animals have the enriching ability to sense the world around them. We sense our surroundings through efficiently designed stimuli sensors, and produce distinctive sensations accordingly. We are equipped with exceptionally sensible, accurate and complete input feature detectors.

However, perhaps one of the most fascinating qualities about a sensorial feeling is that we can mentally encapsulate it as a given configuration and readily identify it at a future similar reocurrence. Our nervous system allows for the formation of a correspondence map between the world outside and the interior reality. The computational processing level and the efficiency of such endeavor continues to excite laborious research in the topic of perceptual neuroscience.

In here, we focus on visual perception. In this case, the system receives the physical image information from the retina and then parallel process the features from the current state of the visual scene. The information signals follow a feedforward pathway and integrations are then carried in hierarchically higher brain areas. At those stages, feedback underlies contextual information, and the neurons can then change their functioning state and produce new conjectures, educated guesses of high success probability about the input's nature.
The unification of these parallel outputs is proposed to finally amount to a conscious sensorial perception.
There is no identity complete copy of the world outside within our percepted reality - Nonetheless, perhaps contrary to our intuition, sensitive sophisticated guesses can be just as good - and biologically exequivable.

%The system comprises peripheral receptors as well as central neurons. From input neurons in the retina that receive the physical image information, signals follow the visual pathway connecting to the brain, where features from the current state of the environment are parallel processed and then hierarchically repeated at higher level stages. In here, based on previously learned information of working logic inferences, the mind makes new conjectures, educated guesses of high success probability about what it is that we could be "seeing", much analogously to an extraordinarily efficient and tuned machine learning algorithm. There's no one-to-one unequivocal complete copy of the world outside within our percepted reality - But, contrary to our intuition, perhaps sensitive sophisticated guesses are just as good - as well as biologically exequivable.
%The unified binding of the parallel results of our cognition processes is presumed to amount to a conscious perception of a given state of the sensorial world.

To investigate this complex scheme, its stages and underlying principles must be described and interpreted. 
The purpose of this project revolves around the surround modulation effect: the finding that, if a stimulus is present in the receptive field (RF), then its surround does influence the resulting signal form of action potentials. 

We propose the study of the spatial structure of surround modulation with motion visual scenes. This means, to analyze the effects of the location of stimuli grading patches, with varied orientations and movement directions, around the RF in the ring-shaped surround of multiple neurons of different orientation \textit{selectivity} (see section 2.2).

Following this objective, recordings of large populations of visual cortex neurons will be obtained using two-photon microscopy in transgenic mice expressing genetically encoded calcium indicators while they observe different sets of flanking stimuli. 

Various multidisciplinary concepts are crucial to conduct this task, and these will be motivated and exposed in the present document.

In the following section \ref{Theoretical}, visual neuroscience main notions are described, to introduce the relevant theoretical concepts. In section \ref{Plan}, the experimental previous methods on the matter of SM are also regarded, motivating this proposal's goals and novelty. The detailed experimental procedures this project aims will then be established as to answer remaining questions about SM. This document then accounts for the relevant experimental and analytical technologies, in section \ref{Experimental}: First, an explanation is provided on behaviour measurement control settings and a programmed stimuli display protocol, as well as the further adaptations it shall undergo for this project. Secondly, the relevant principles in chemical indicators and optical physics mechanisms to use for the expression and detection of neural activity will be clarified. Finally, the analysis of the experimental data will require appropriate methods for the treatment of a large number of individual cells - We will present a possible working approach to this, Suit2p. This report will then follow with a commented bibliography, section \ref{Bib}, the calendarization of the thesis work, section \ref{Cal}, and a final discussion of the general project's plan overview, section \ref{Con}.

%\\In the following state of the art section, as a theoretical introduction, we will start by presenting visual neuroscience main general notions (section 2.1), and then specify the current understanding of the retina's neurophysiology (section 2.2) and visual cortex (section 2.3). We will then enunciate the perception organization central ideas (section 2.4) and finally review the modern surround modulation investigation results, particularizing for V1 (section 2.5).

%%other principles in neurobiology are relevant, as well as computation familiarity with Matlab, $C^{++}$, specific behavior control and measurement packages and extensions, as well as optical physics and instrumentation engineering fundaments. These will be delineated
%\\The experimental phases of this project, will be delineated starting with a section for the stimuli and software development programming (section 3.1), a review on the sort of chemical indicators, genetic manipulation and physical mechanisms used for this experiment's expression of neural activity (section 3.2) and finalized by a presentation of the recording optical technology, two-photon excitation microscopy (section 3.3).

%\\Finally, the analysis of the experimental data will require appropriate methods for the treatment of great amounts of possible parameter configurations, each for a large number of individual cells - We will present a possible working approach to this, Suit2p (section 4).

%This report will follow with a commented bibliography (section 5), the calendarization of the to-be-done work (section 6) and a final discussion of the general project's plan overview (section 7).