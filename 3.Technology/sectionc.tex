\section{Two-photon laser scanning microscopy}
\label{sec:sectionc}

Neuronal phenomena can be relevant in broad scale ranges, both spatially and temporally. With high resolution, sensitivity, contrast and being able to track events over large cortical ranges, two-photon excitation (2PE) laser scanning microscopy provides a way of accessing fluorescent objects, such as the GCaMP6 expressing neurons, by selectively exciting them and detecting the produced light signal. This technique can furthermore be applied to living or intact tissue, with minimal photodamage (phototoxicity and photobleaching). The probability of detecting a signal photon per excitation event is greater than with previous techniques, especially for imaging deep in scattering tissue.

Two low-energy photons (deep red and near IR) are sent by a focused laser to a fluorophore unit and excite, in combination, the higher-energy electronic transition required for the emission of fluorescent light. 2PE is a nonlinear process, as the absorption rate depends on the squared value of light intensity. This intensity drops quadratically with the distance from the focus. Therefore, if the numerical aperture objective is small enough, this excitation is localized, and the excitation can affect a very small focal volume and produce good 3D contrast and resolution.

Another advantage of 2PE is in its relation with scattering. As photons enter tissue, they scatter and deviate their paths according to inhomogeneities found in the refractive index of the medium, and this reduces the amount of light delivered to the focus and from the fluorescent molecule to the detection apparatus. However, 2PE uses near IR beams that penetrate tissue better than visible waves (as used in one-photon microscopy); the non-linearity of 2PE also contributes to the reduced scattered photons; and finally, since the excitation is localized, all the fluorescence photons coming from the excited molecule, even if scattered, portray useful signal, not being lost and not contributing to background noise.

Typically, a laser is focused on a given target plane and scanned over the sample. When an excitation occurs, it is detected by photodetectors. These responses are then summed over some microseconds and mapped to single pixels of an image.

The lasers used for this method should be powerful enough to compensate for the small two-photon cross-sections and produce sufficient signal levels. Moreover, 2PE efficiency increases as the inverse of pulse duration, and the device should thus be suitable for short light pulses. Mode-locked Ytterbium-doped and Cr:forsterite lasers suffice these requirements for the considered wavelengths.

Objectives are used for the essential focusing of the laser beams. Another important consideration regards the detectors: These should cover large sensitive areas (millimeters), as well as contemplate good gain, quantum efficiency, and other important thresholds. Photomultiplier tubes (PMTs) are usually applied for this purpose.